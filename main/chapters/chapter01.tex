\chapter{The Real Numbers}
\section{Introductory Proofs}
% squareroot of 2 is irrational
\begin{theorem}
There is no rational number whose square is 2 \\
\textit{Proof.} A rational number can be expressed as $ p/q $ where both $ p $ and $ q $ are integers. We can prove this using contradiction. If there does exist a rational number whose square is 2, then $$ \left(\dfrac{p}{q}\right)^2 = 2$$ We can rearrange this to find that \[ p^2 = 2 q^2 \] Based on this, we know that $ p^2 $ is even. We must now show that since $ p^2 $ is even $ p $ must also be even. Here we can say that since $ p^2 = p * p $ and $ p^2 $ is divisible by 2, $ p $ must also be divisible by two, since $ p $ is an integer. We can now introduce another variable $ a $ to say $ 2a = p $ so \[ \dfrac{4a^2}{q^2} = 2 \] rearranging we find 
\[ \dfrac{q^2}{a^2} = 2 \] now using the previous argument we know that $ q $ is also even. This now tells us that $ p/q $ is further divisible and thus does not define a rational number.  
\end{theorem} 																		 						
\section{Some Preliminaries}
The necessary vocabulary for Analysis comes from set theory. This section will just be a quick overview. \\
\subsection{Sets}
A \textit{set} is any collection of objects. These objects are referred to as \textit{elements} of the set. \\
Given a set \textit{A}, we write $ x \in A $ if $ x $ is an element of $ A $. If $ x $ is not an element of $ A $, then we write $ x \notin A $. Given two sets $ A $ and $ B $, the \textit{union} is written $ A \cup B $ and is defined by asserting that. 
\begin{center} 
	$ x \in A \cup B $ provided that $ x \in A $ or $ x \in B $. 
\end{center}
The \textit{intersection} $ A \cap B $ is the set defined by the rule. 
\[ x \in A \cap B \text{ provided } x \in A \text{ and } x \in B \]
\begin{example}
	Let
	\begin{align*}
		A_1 &= N = {1,2,3,...}, \\
		A_2 &= {2,3,4,...}, \\
		A_3 &= {3,4,5,...},
	\end{align*}
	and, in general, for each $ n \in N $, define the set
	\[ A_n = {n, n+1, n+2, ...}. \]
	The result is a nested chain of sets
	\[ A_1 \supseteq A_2 \supseteq A_3 \supseteq ... , \]
	where each successive set is a subset of all the previous ones. Notationally, 
	\[ \bigcup^\infty_{n=1} A_n = A_1. \]
\end{example}
Given $ A \subseteq \textbf{R} $ the compliment of $ A $, written $ A^c $, refers to the set of all elements of \textbf{R} not in $ A $. Thus, for $ A \subseteq \textbf{R} $, 
\[ A^c = {x \in \textbf{R} : x \notin A}. \] \\
Here are some notation for sets:
\[ \textbf{N} = \{1,2,3,...\} \]
In words this set is the collection of all natural numbers. 
\subsection{Functions}
Given two sets $ A $ and $ B $ a function from $ A $ to $ B $ is a rule or mapping that takes each element $ x \in A $  and associates with it a single element of $ B $. 
\begin{example}
	\[ g(x) = 
		\begin{cases}
			1 \mbox{ if } x \in \textbf{Q} \\
			0 \mbox{ if } x \notin \textbf{Q}
		\end{cases} 
	\]
	The domain of $ g $ is all of $ \textbf{R} $ but the range is only 0 or 1. There is no single formula for this function, but it does qualify as a function.
\end{example}
\begin{example}[Triangle Inequality]
	The absolute value function is also a function, and has been given its own symbol. 
	\[ 
	|x| =
		\begin{cases}
			x \mbox{ if } x \geq 0 \\
			-x \mbox{ if } x < 0
		\end{cases}
	 \]
\end{example}
\subsection{Logic and Proofs}
Using an indirect proof when a direct proof is available is not the wave. 
\begin{theorem}
	Two real numbers $ a $ and $ b $ are equal if and only if for every real number $ \epsilon > 0$ if follows that $ |a-b| < \epsilon $ \\
	\textit{Proof.} Two important parts of this theorem. "if and only if" in math means that the statement must be true in both directions. In the forward it is quite easy to see: if $ a = b $ then $ |a-b| $ is indeed less than all real numbers. \\
	Now to prove the converse statement. We can use a proof by contradiction. We arrive at this approach because of the "for every" in the theorem which suggests that we only need to show one example where this does not work. \\
	The choice of
	\[ \epsilon_0 = |a-b| > 0 \]
	contradicts our previous statement so 
	\begin{gather}
		|a-b| < \epsilon_0 \mbox{ and } |a-b|= \epsilon_0
	\end{gather}
	cannot both be true. This contradiction means that the initial assumption \[ a \neq b \] does not hold. 
\end{theorem}
\subsection{Induction}
The main principle behind induction is that if $ S $ is some subset of $ \textbf{N} $ with the property that
\begin{enumerate}
	\item $ S $ contains 1 and
	\item whenever $ S $ contains a natural number $ n $, it also contains $ n + 1 $,
\end{enumerate}
then it must be that $ S = \textbf{N} $. 
\begin{example}
	Let $ x_1 = 1 $, and for each $ n \in \textbf{N} $ define \[ x_{n+1} = (1/2)x_n + 1 \]
	Using this rule we can compute the series $ 3/2, 7/4, ... $. Lets use induction to prove 
	\[ x_n \leq x_{n+1} \] 
	for all values of $ n \in \textbf{N} $. \\
	if we have $ x_n \leq x_{n+1} $, then it follows that $ x_{n+1} \leq x_{n+2} $. \\
	Multiplying by $ 1/2 $ and adding one gives us 
	\[ \dfrac{1}{2} x_n + 1 \leq \dfrac{1}{2} x_{n+1} + 1, \]
	which substituting from the given statements gives us $ x_{n+1} \leq x_{n+2} $, and we can continue this for all following values. By induction this is proved for all $ x \in \textbf{N} $. 
\end{example}
\section{The Axiom of Completeness}
\subsection{An Initial Defintion for \R}
First, \R is a set containing \Q. The operations of addition and multiplication on \Q extend to all of \R. We assume that \R is an \textit{ordered field}, which contains \Q as a subfield. \\
The most distinctive assumption about the real number system is that \R does not contain gaps. \\
\textbf{Axiom of Completeness. }\textit{Every nonempty set of real numbers that is bounded above has a least upper bound.} \\
\subsection{Least Upper Bounds and Greatest Lower Bounds}
A real number $ s $ is the \textbf{supremum}, least upper bound, for a set $ A \subseteq \textbf{R} $ if it meets the following two criteria
\begin{enumerate}
	\item $ s $ is an upper bound for $ A $;
	\item if $ b $ is any upper bound for $ A $, then $ s \leq b $.
\end{enumerate}
Use notation $ s = \sup A $ for least upper bound. \\
Similarly, there is the greatest lower bound called \textbf{infimum}. \\
A maximum or minimum follows the usual definition, but it must be in the set. Thus, supremum and infimum are different ideas. \\
Although a set can have a host of upper bounds, it can have only one least upper bound. \\
\putpic{sup}
\begin{example}
	Let
	\[ A = \left\{ \frac { 1 } { n } : n \in N \right\} = \left\{ 1 , \frac { 1 } { 2 } , \frac { 1 } { 3 } , \ldots \right\} \]
	The set $ A $ is bounded above and below. Successful candidates for an upper bound include $ 3, 2, 3/2 $. For the least upper bound, we claim $ \sup A = 1 $. To argue this rigorously, we need to verify that properties (i) and (ii) hold. For (i), we just observe that $ 1 \geq 1/n $ for all choices of $ n \in \N $.\\
	To verify (ii), we begin by assuming we are in possession of some other upper bound $ b $. Because $ 1 \in A $ and $ b $ is an upper bound for $ A $, we must have $ 1 \leq b $. This is precisely what property (ii) asks us to show. \\
	Although we do not quite have the tools we need for a rigorous proof, it should be somewhat apparent that $\inf A = 0 $. \\
	An important lesson is that $ \sup A $ and $ \inf A $ may or may not be elements of the set $ A $. This issue is tied to understanding the	crucial difference between the maximum and the supremum (or the minimum
	and the infimum) of a given set.
\end{example}
\begin{lemma}
	Assume $ s \in \R $ is an upper bound for a set $ A \subseteq \R $. Then, $ s = \sup A $ \myiff, for every choice of $ \epsilon > 0 $, there exists an element $ a \in A $ satisfying $ s-\e < a $
\end{lemma}
\section{Consequence of Completeness}
The first application of the AoC is to mathematically express that the real line contain no gaps.
\subsection{Nested Interval Property}
\begin{theorem}
	For each $ n \in \textbf{N} $, assume we are given a closed interval $ I_n = [a_n, b_n] = {x \in \textbf{R}: a_n \leq x \leq b_n}$. Assume also that each $ I_n $ contains $ I_{n+1} $. Then, the resulting nested sequence of closed intervals
	\[ I_1 \subseteq I_2 \subseteq I_3 \subseteq ... \]
	has a nonempty intersection; that is, $ \cap_{n =1}^\infty I_n \neq \emptyset$.
	Define a new set $ A = {a_n : n \in \textbf{N}} $ set $ x = \sup A $ so $ x > a_n $ but since it is the least upper bound $ x \leq b_n $. So, \[ a_n \leq x \leq b_n \]
\end{theorem}
\subsection{Density of $ \Q $ in $ \R $}
\begin{theorem}[Archimedean Property]
	The theorem follows
	\begin{enumerate}
		\item Given any number $ x \in \R$, there exists an $ n \in \N $ satisfying $ n > x $. 
		\item Given any real number $ y > 0 $, there exists an $ n \in \N $ satisfying $ 1/n < y $.
	\end{enumerate}
	For the first part lets use an indirect proof, and say that $ \N $ is bounded, therefore it must have $ \alpha = \sup N $. Now if we have the value $ \alpha - 1 $, this is no longer the upper bound so we have $ \alpha -1 < n $ where $ n \in \N $. We then have $ \alpha < n + 1 $, but $ n+1 \in \N $, therefore no supremum exists. 
\end{theorem}
\begin{theorem}[Density of $ \Q $ in $ \R $]
	For every real numbers $ a $ and $ b $ with $ a < b $, there exists a rational number $ r $ satisfying $ a < r < b $. \\
	\textit{Proof.} To simplify, let's assume $ 0 \leq a < b $, you can revisit the assumption easily. A rational number can be expressed as follows for $ m, n \in \N $ so that \[ a < \dfrac{m}{n} < b\] Using the previous theorem we pick $ n \in \N $ so that \[ \dfrac{1}{n} < b -a \] Multiplying the first inequality gives us \[ an < m < bn \] Now we must pick $ m $ such that \[ m -1 \leq na < m \] this immediately yields the first half \[ a < \dfrac{m}{n} \] Using the previous result \[ b > a + \dfrac{1}{n} \] we can write 
	\[ m \leq na + 1 \]
	\[ m < n \left(b - \dfrac{1}{n}\right) + 1 \]
	\[ m = nb \] and then we arrive at the desired 
	\[ a < m/n <b \]
\end{theorem}
\subsection{The Existence of Square Roots}
\begin{theorem}
	There exists a real number $ \alpha \in \R $ satisfying $ \alpha^2 = 2 $
	\textit{Proof.} Consider the set 
	\[ T = {t \in \R : t^2 < 2} \]
	and set $ \alpha = \sup T $. We can prove that $ \alpha^2  = 2$ by ruling out the possibilities $ \alpha^2 < 2 $ and $ \alpha^2 > 2 $. Since, $ \alpha $ is the supremum, $ \alpha^2 < 2 $ we need to show violates that it is an upper bound, and $ \alpha^2 > 2 $ violates that it is the least upper bound. \\
	First lets rule out $ \alpha ^2 < 2 $, we need to show that there is a $ t $ that larger than $ \alpha $, so consider \\
	\begin{align*}
		\left(\alpha + \dfrac{1}{n}\right)^2 &= \alpha^2 + \dfrac{2 \alpha}{n} + \dfrac{1}{n^2} 
		\intertext{Keep in mind that $ n \in \N $}
		 \alpha^2 + \dfrac{2 \alpha}{n} + \dfrac{1}{n^2}  < \alpha^2 + \dfrac{2 \alpha}{n} + \dfrac{1}{n} \\
		 = \alpha^2 + \dfrac{2 \alpha + 1}{n}
	\end{align*}
	Okay, so up until now we haven't actually used our assumption that $ \alpha^2 < 2 $. Due to this assumption we know the value $ 2 - \alpha ^2 > 0 $, so for a $ n_0 \in \N $ we can find
	\[ \dfrac{1}{n_0} < \dfrac{2 - \alpha ^2}{2 \alpha + 1} \] 
	This implies $ (2 \alpha + 1)/n_0 < 2 - \alpha ^2 $. Since, it is greater than the value already present we can make the substitution in the expression and maintain the inequality. 
	\[ \left(\alpha + \dfrac{1}{n_0}\right)^2 < \alpha^2 + (2- \alpha^2) = 2\]
	We arrive at a contradiction, because a value is greater than $ \alpha $ which is the upper bound $ \alpha^2 < 2 $ cannot happen. \\
	Now lets assume that $ \alpha ^2 > 2 $. This time we use 
	\[ \left(\alpha - \dfrac{1}{n}\right)^2 = \alpha^2 - \dfrac{2\alpha}{n} + \dfrac{1}{n^2} \]
	\[ > \alpha^2 - \dfrac{2\alpha}{n} \]
	Now we can use our assumption, $ \alpha ^2 > 2 $, to find a $ n_0 \in \N $ such that 
	\[ \dfrac{1}{n_0} < \dfrac{\alpha^2 -2}{2 \alpha} \]
	\[ \dfrac{2 \alpha}{n_0} < \alpha^2 - 2 \]
	\[ -\dfrac{2 \alpha}{n_0} > 2 - \alpha^2 \]
	Now we can make the substitution and get 
	\[ \left(\alpha - \dfrac{1}{n}\right)^2 > \alpha^2 + 2 - \alpha^2 = 2  \]
	Now that we have ruled out $ > 2 $ and $ < 2 $ we know that $ \alpha^2 = 2 $.
\end{theorem}

\section{Countable and Uncountable Sets}
\subsection{Cardinality}
The term  refers to the size of a set. 
\indent Two sets $ A $ and $ B $ have the same cardinality if there exists $ f: A \rightarrow B $ that is $ 1-1 $ and onto. In this case, we write $ A \sim B $\\

\begin{example}
	If we let $ E = {2,4,6,...} $ bet the set of even natural numbers, then we can show $ \N \sim E $. Let $ f: \N \rightarrow E $ be given by $ f(n) = 2n $. It is true that $ E $ is a subset of $ \N $, but this does not mean that $ E $ is "smaller" than $ \N $. This is a bias based on an exposure to finite sets, The definition of cardinality is quite specific, and from this point of view $ E \sim \N $.
\end{example}
\subsection{Countable Sets}
A set $ A $ is countable if $ \N \sim A $. An infinite set that is not countable is called uncountable. 
\begin{theorem}
	(i)The set \Q is countable. (ii) The set \R is countable. 
\end{theorem}
\section{Cantor's Theorem}
\begin{theorem}
	The open interval \[ ( 0,1 ) = \{ x \in \mathbf { R } : 0 < x < 1 \} \] is uncountable. \\
	Proof by contradiction,  assume function $ f:\N \rightarrow (0,1) $ is 1-1 and onto. \\
	Say that $ m, n \in \N $ is the digit from the set $ \{0,1,2,...,9\} $ \\
	\putpic{coolproof}
	Now say we want to define a number $ x \in (0,1) $ such that $ x = .b_1b_2b_3... $ using this rule
	\[ b _ { n } = \left\{ \begin{array} { l l } { 2 } & { \text { if } a _ { n n } \neq 2 } \\ { 3 } & { \text { if } a _ { n n } = 2 } \end{array} \right. \]
	Work through this and see how you will go through all of \N before finding the number. 
\end{theorem}
\section{Power Sets}
Given a set $ A $, the power set $ P(A) $ refers to the collection of all subsets of $ A $. It
is important to understand that$  P (A) $ is itself considered a set whose elements are the different possible subsets of $ A $.
\begin{theorem}[Cantor's Theorem]
	Given any set $ A $, there does not exist a function $ f:A \rightarrow P(A) $ that is onto. 
\end{theorem}
\section{Exercises}
\begin{q}
	Prove the $ \sqrt{2}$ is not rational.
\end{q}
\begin{q}
	Prove that the product of two odd integers is also odd.
\end{q}
\begin{q}
	Let $ A = \{ a , b , c \} $. List the elements of $ P(A) $.
\end{q}
\begin{q}
	Use the Archimedeam Property of \textbf{R} to prove that
	\[ \inf \{ 1 / n : n \in \mathbf { N } \} = 0 \]
\end{q}
\begin{q}
	Show that the set of all finite subsets of \textbf{N} is countable set.
\end{q}

































